%Example of use of oxmathproblems latex class for problem sheets
\documentclass{oxmathproblems}

%(un)comment this line to enable/disable output of any solutions in the file
%\printanswers

%define the page header/title info
\trinityterm{STU33009}
\course{Statistical Methods for Computer Science}
\assignmentnumber{1}

% add further contact details to footer if desired,
%e.g. email address, or name and email address
\contact{Hamza Mughees, 17329860}


\begin{document}

\begin{questions}

\miquestion
\begin{parts}
  \part For the selection of the first letter, we would have 10 choices. After selecting this letter, we would now have 9 choices for the second letter, of which each choice can be matched with 1 of the 10 choices made for the first letter ($10\cdot 9$). This pattern would continue until we have used up all 10 letters. Therefore:
  $$10\cdot 9\cdot 8\cdot 7\cdot 6\cdot 5\cdot 4\cdot 3\cdot 2\cdot 1$$
  $$10!=3628800$$
  \part Since the restriction imposes that E and F must be next to each other, we can, at first, treat them as one:
  $$9!=362880$$
  However, since the E and the F can be in any order, we must multiply this quantity by 2:
  $$362880\cdot 2=725760$$
  \part We take the factorial of the total number of letters. We divide this by the factorial of the occurrence of each letter in the word (ignoring the letters that appear only once):
  $$\frac{6!}{3!\cdot 2!}=60$$
  \part We would calculate the total number of ways 3 letters can be chosen from 5:
  $${5\choose 3}=10$$
\end{parts}

\miquestion
\begin{parts}
  \part If we roll 2 six-sided dice, each outcome from dice 1 can be paired with all of the outcomes from dice 2. Therefore, the total outcomes would be $6\cdot 6=6^2=36$. For $n$ dice rolls, the total number of outcomes would be $6^n$. For 4 dice rolls:
  $$6^4=1296$$
  \part If exactly two dice roll a 3, the remaining two only have 5 options to roll from, as opposed to 6, since they mustn't be 3. There are two remaining dice:
  $$5^2=25$$
  We must now multiply this by the number of ways the 2 dice that roll the 3's, can be chosen from 4:
  $$25\cdot {4\choose 2}=25\cdot 6=150$$
  \pagebreak
  \part From part (b), we know that the number of ways to obtain exactly two 3's is 150. We also know that the number of ways to obtain exactly four 3's is, of course, 1. What's remaining is to calculate the number of ways to obtain exactly three 3's. 
  \linebreak 
  \linebreak 
  If we have exactly three dice rolling a 3, the remaining die has only 5 numbers to roll from, since it mustn't roll a 3. However, since three out of the four dice roll 3's, the remaining die can be any 1 of the 4 dice. Therefore, we must multiply the possible outcomes of the remaining die, 5, by the number of possible dice that can become the remaining die, which would be ${4\choose 1}$:
  $${4\choose 1}=4$$
  $$5\cdot 4=20$$
  Now we can simply add up all of these possible outcomes to get our final answer:
  $$150+20+1=171$$
\end{parts}

%force a page break for better layout of questions 
%NOTE: only force pagebreaks at the final stage for perfecting the layout
%\newpage

\miquestion
\begin{parts}
  \part We have 8 cards, containing 2 cards from each of the 4 suits:
  $$\frac{8!}{2!\cdot 2!\cdot 2!\cdot 2!}=\frac{40320}{16}=2520$$
  \part There are 4 ways in which a card can be dealt with its duplicate from the other deck. In order to calculate this, we ignore the 4 duplicates from the set of 8 cards, and calculate the number of ways in which 2 cards can be chosen from the remaining 4:
  $${4\choose 2}=6$$
  \part There are a total of 4 ``good'' cards in the set of 8. There are two ways to deal two duplicate ``good'' cards out of the set of ``good'' cards. Using the same logic as in the previous part with, the addition of another step, we can ignore the duplicate scenarios, and calculate the number of ways to select 2 cards from the remaining 2 ``good'' cards:
  $${2\choose 2}=1$$
  We can now add this quantity to the number of ways to deal 2 duplicates, which is 2:
  $$2+1=3$$
\end{parts}

\end{questions}

\end{document}
