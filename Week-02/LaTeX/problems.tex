%Example of use of oxmathproblems latex class for problem sheets
\documentclass{oxmathproblems}

%(un)comment this line to enable/disable output of any solutions in the file
%\printanswers

%define the page header/title info
\trinityterm{STU33009}
\course{Statistical Methods for Computer Science}
\assignmentnumber{2}

% add further contact details to footer if desired,
%e.g. email address, or name and email address
\contact{Hamza Mughees, 17329860}


\begin{document}

\begin{questions}

\miquestion
\begin{parts}
    \part The sample space is the set of all possible outcomes. A six-sided die has a sample space of 6 elements. A six-sided die rolled three times would have a sample space of size $6\cdot6\cdot6=6^3$:
    $$6^3=216$$
    \part In order to calculate this, we can first calculate the number of outcomes in which no 2 is rolled. We can now only pick from 5 possibilities in each of the three rolls:
    $$5^3=125$$
    We can now subtract this quantity from the total size of the sample space to get the number of outcomes in which at least a 2 is rolled:
    $$216-125=91$$
    The probability that at least one 2 is rolled is the following:
    $$\frac{91}{216}=0.4212962963$$
    \part
    \begin{verbatim}
iters = 100000;
outcomes = randi([1,6],[1,3,iters]);
count = 0;

for i = 1:iters
    if any(outcomes(:,:,i) == 2)
        count = count + 1;
    end
end

probability = count/iters

>>
Probability =
    0.4222
    \end{verbatim}
    \part If all of the rolls resulted in a 6, the sum would be 18. Since we want the sum to be 17, a single roll must be a 5 and the rest 6. However, this 5 could be outcome of any one of the three rolls:
    $${1\choose 3}=3$$
    Therefore, there are 3 possible outcomes in which the sum can be 17. The probability of one of these outcomes occurring would be:
    $$\frac{3}{216}=0.01388888889$$
    \part We can calculate this by calculating the probability that the remaining die rolls add up to 11. There are two remaining die rolls, therefore the sample space now becomes:
    $$6^2=36$$
    In order for 2 die rolls to add up to an 11, one of the die rolls should be a 5, there are 2 ways this can happen. Therefore, the probability that the three rolls sum to 12 given that the first roll is a 1 would be:
    $$\frac{2}{36}=0.05555555556$$
\end{parts}

\miquestion
\begin{parts}
    \part There are two ways in which the second die can roll a 5. In one case, the first die could roll a 1, which would result in a second six-sided die to be rolled, which can roll a 5. The probability of the first die rolling a 1 would be $\frac{1}{6}$ and the probability of the second die rolling a 5 would also be $\frac{1}{6}$. The probability of this case would be:
    $$\frac{1}{6}\cdot\frac{1}{6}=\frac{1}{36}$$
    In the second case, the fist die could roll a number other than 1. The probability of this would be $\frac{5}{6}$. Hence, the subsequent throw would be of a 20-sided die which would need to roll a 5. The probability of this happening would be $\frac{1}{20}$. The probability of this case would be:
    $$\frac{5}{6}\cdot\frac{1}{20}=\frac{5}{120}=\frac{1}{24}$$
    We can then add the probability of these cases to obtain the probability of any of them happening:
    $$\frac{1}{36}+\frac{1}{24}=\frac{5}{72}=0.06944444444$$
    \part In order for the second roll to be a 15, the first roll mustn't be a 1. the probability of the first roll not being a 1 is $\frac{5}{6}$. Subsequently, the probability of the second roll being a 15 is $\frac{1}{20}$. The probability of both happening would be:
    $$\frac{5}{6}\cdot\frac{1}{20}=\frac{5}{120}=\frac{1}{24}=0.04166666667$$
\end{parts}

\miquestion
Let $E$ be the event that the suspect is guilty.
\newline
Let $F$ be the event that the suspect possesses the found characteristic.
\newline
\newline
We know that the probability of the suspect being guilty is 0.6, (i.e. $P(E)=0.6$). We also know that the probability that a person from the general population, including a suspect that isn't guilty, possesses the characteristic is 0.2, (i.e. $P(F|E^c)=0.2$). We also know that the probability that a guilty suspect possesses the characteristic is 1, (i.e. $P(F|E)=1$). 
\newline
\newline
Using Bayes' Rule, we get:
$$P(E|F)=\frac{P(F|E)P(E)}{P(F)}=\frac{P(F|E)P(E)}{P(F|E)P(E)+P(F|E^c)P(E^c)}$$
$$P(E|F)=\frac{1\cdot 0.6}{(1\cdot 0.6)+(0.2\cdot 0.4)}=0.882353$$

\miquestion
Let $E_i$ be the probability that the user is in cell $i$.
\newline
Let $F$ be the probability of observing two bars from the cell tower
\newline
Let $n$ be the number of cells in the map.
\newline
\newline
We know that the probability that the user is in cell $i$ before observing a signal from the cell tower is 0.05, (i.e. $P(E_i)=0.05$). We also know that the probability of observing a signal from the cell tower given that the user is in cell $i$ is 0.75, (i.e. $P(F|E_i)=0.75$).
\newline
\newline
Bayes' Rule tells us the following:
$$P(E_i|F)=\frac{P(F|E_i)P(E_i)}{P(F)}$$
We can calculate P(F) using marginalisation. The probability of observing a signal from the cell tower would be equal to the sum of the probabilities of observing a signal at each cell $i$. From this we get:
$$P(F)=\sum_{i=0}^{n}P(F|E_i)P(E_i)$$
\newpage
Plugging in the values, we get:
\begin{align*}
    P(F)=(0.75)(0.05)+(0.95)(0.1)+(0.75)(0.05)+(0.05)(0.05)\\
        +(0.05)(0.05)+(0.75)(0.1)+(0.95)(0.05)+(0.75)(0.05)\\
        +(0.01)(0.05)+(0.05)(0.05)+(0.75)(0.1)+(0.95)(0.05)\\
        +(0.01)(0.05)+(0.01)(0.05)+(0.05)(0.1)+(0.75)(0.05)\\
        =0.504
\end{align*}
Plugging each cell into the formula, we can form the following matrix:
$$
\begin{matrix}
    \frac{(0.75)(0.05)}{0.504}&\frac{(0.95)(0.1)}{0.504}&\frac{(0.75)(0.05)}{0.504}&\frac{(0.05)(0.05)}{0.504}\\
    \frac{(0.05)(0.05)}{0.504}&\frac{(0.75)(0.1)}{0.504}&\frac{(0.95)(0.05)}{0.504}&\frac{(0.75)(0.05)}{0.504}\\
    \frac{(0.01)(0.05)}{0.504}&\frac{(0.05)(0.05)}{0.504}&\frac{(0.75)(0.1)}{0.504}&\frac{(0.95)(0.05)}{0.504}\\
    \frac{(0.01)(0.05)}{0.504}&\frac{(0.01)(0.05)}{0.504}&\frac{(0.05)(0.1)}{0.504}&\frac{(0.75)(0.05)}{0.504}\\
\end{matrix}
$$
This would simplify to:
$$
\begin{matrix}
    0.0744&0.1885&0.0744&0.0050\\
    0.0050&0.1488&0.0942&0.0744\\
    0.0010&0.0050&0.1488&0.0942\\
    0.0010&0.0010&0.0099&0.0744\\
\end{matrix}
$$
Here is the Matlab program:
\begin{verbatim}
P_Es = [
    0.05 0.1 0.05 0.05;
    0.05 0.1 0.05 0.05;
    0.05 0.05 0.1 0.05;
    0.05 0.05 0.1 0.05;
];

P_F_given_Es = [
    0.75 0.95 0.75 0.05;
    0.05 0.75 0.95 0.75;
    0.01 0.05 0.75 0.95;
    0.01 0.01 0.05 0.75;
];

products = P_F_given_Es.*P_Es;
P_F = sum(products, 'all');
P_Es_given_F = products/P_F

>>
P_Es_given_F =
    0.0744    0.1885    0.0744    0.0050
    0.0050    0.1488    0.0942    0.0744
    0.0010    0.0050    0.1488    0.0942
    0.0010    0.0010    0.0099    0.0744
\end{verbatim}

\end{questions}

\end{document}
